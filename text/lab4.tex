% Use xelatex
% chktex-file 24
\documentclass{hse_document}

\begin{document}

\maketoc{ПРОЕКТИРОВАНИЕ ПРОГРАММНЫХ ИНТЕРФЕЙСОВ}{Отчёт по лабораторной работе}{В.П. Куприн}

\tableofcontents
\clearpage

\addcontentsline{toc}{section}{Введение}
\section*{Введение} \label{Введение}

\textbf{Название работы:}
Лабораторная работа №4 <<Проектирование программных интерфейсов>>

\textbf{Цель работы:}
Научиться проектировать систему взаимодействий между компонентами и слоями программной системы.

\textbf{Ожидаемый результат:}
Описание API компонентов программной системы с использованием подходящих
инструментов (диаграмма классов (CLASS), XML, SWAGGER, таблицы спецификаций и
т.д.)

\section{Проектирование программных интерфейсов}\label{sec:interfaces}

Поскольку разрабатываемое приложение является API, то при проектировании опишем маршруты и данные,
с которыми оно работает.

Результатом данного этапа работы является Swagger-спецификация системы, доступная по \href{https://app.swaggerhub.com/apis/frogix/workflow-engine-api/1.0.0}{ссылке}.

\subsection{Описание данных}

Было принято решение <<общение>> клиента с сервером определить обменом JSON-объектами: этот формат более лаконичен.

При создании машины состояния клиент передаёт объект, содержащий следующие данные:

\begin{compactenum}
\item Массив возможных состояний;
\item Массив разрешённых переходов между состояниями;
\item Массив триггеров переходов по событию;
\item Набор триггеров переходов по условию;
\item Идентификатор начального состояния.
\item Объект с полями и их значениями.
\end{compactenum}

Рассмотрим более детально, что из себя представляет каждое из полей этого объекта.

Состояние определяется, как объект, у которого, есть идентификатор и название.

Переход задаётся парой начальное состояние --- конечное состояние.

Триггер перехода содержит название события и идентификатор нового состояния.

Информация об условном переходе включает в себя имя поля объекта для сравнения, условный оператор, значение для сравнения и идентификатор нового состояния.

\subsection{Описание маршрутов}

Все маршруты для машины состояний будут начинаться с <</machine>>.

POST-запрос на этот узел отвечает за создание машины, все параметры передаются через JSON-тело запроса.

GET-запрос к <</machine>> для отладки было решено использовать для получения списка всех доступных машин состояний.

Запросы к конкретным машинам осуществляются с помощью указания их идентификаторов в URL запроса <</machine/{machineId}>>. Так, GET-запрос к этому маршруту в качестве ответа получает информацию о запрошенной машине состояний.

Для передачи изменений, событий, получения состояний используются пути <</change>>, <</event>>, <</state>>, добавленные к <</machine/{machineId}>>.

В спецификации были определены коды ошибок при некорректных данных, недопустимых переходах.

\end{document}
