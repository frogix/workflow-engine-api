% Use xelatex
% chktex-file 24
\documentclass{hse_document}

\begin{document}

\maketoc{ПРОЕКТИРОВАНИЕ И РЕАЛИЗАЦИЯ ПРОГРАММНОЙ СИСТЕМЫ}{Отчёт по лабораторной работе}{В.П. Куприн}

\tableofcontents
\clearpage

\addcontentsline{toc}{section}{Введение}
\section*{Введение} \label{Введение}

\textbf{Название работы:}
Лабораторная работа №5 <<Проектирование и реализация программной системы>>

\textbf{Цель работы:}
Научиться выбирать технологии для реализации программной системы, видеть возможности для применения шаблонов проектирования.

Научиться реализовывать проект программной системы, использовать шаблоны
проектирования при непосредственной реализации системы.

\textbf{Ожидаемый результат:}

Описание проекта реализации программной системы с использованием необходимых средств
представления (диаграмма классов (CLASS), диаграмма последовательностей (SEQUENCE),
модель данных IDEF1x и т.д.).

Реализация программной системы, соответствующая требованиям, сформулированным в
лабораторной работе №1, в соответствии с проектными решениями, выработанными в других
лабораторных работах.

\section{Описание проекта реализации программной системы}\label{sec:project}

При реализации программной системы должны быть разработаны следующие элементы:

\begin{compactenum}
\item Сервер, содержащий свою конфигурацию и обработчик ошибок.
\item Модуль конфигурации приложения, загружающий данные из переменных окружения.
\item Маршрутизатор, вызывающий контроллеры при обработке запросов.
\item Контроллер машины состояний, обращающийся к модели данных для проведения операций над машиной состояний.
\item Модель машины состояний, содержащая схему данных и набор методов.
\item Класс API-ошибки.
\end{compactenum}

В качестве технологий реализации было принято решение выбрать Express.js для обработки HTTP-запросов, MongoDB в качестве базы данных и Mongoose --- клиент к этой базе данных, позволяющий описывать схемы документов (объектов). При выборе стека технологий учитывались следующие параметры: уровень владения технологиями, возможность бесплатного развёртывания, возможность изменения схемы документов по ходу разработки.


\section{Реализации программной системы}\label{sec:implementation}

Во время реализации программной системы учитывались результаты предыдущих работ. В частности, для проверки необходимого функционала использовалась use-case диаграмма, для реализации маршрутов и схем объектов --- спецификация Swagger, ER-диаграмма.


Для обработки ошибок API был реализован соответствующий класс, который хранит HTTP-код ответа и строка сообщения об ошибке. При запуске в режиме Development этот класс также добавляет отладочную информацию.

Для проверки работы API использовался Postman, хотя для демонстрации также возможно использование Swagger.

При реализации системы были обнаружены некоторые неточности в формулировках предыдущих лабораторных работ. Так потребовалось исправить Swagger-спецификацию для обработки изменений нескольких полей объектов (а не одного), поскольку нелогично изменение каждого обновлённого поля обрабатывать в отдельном запросе, если на практике такие изменения могут происходить одновременно, а не только последовательно.

Исходный код программы был загружен на Github (\href{https://github.com/frogix/workflow-engine-api}{ссылка}).

В файле README были коротко описаны основные функции системы: реализованные маршруты и данные, с которыми они работают.

Приложение было опубликовано на сервисе Heroku, который позволяет бесплатно запускать приложения Node.js на своих серверах. Heroku допускает интеграцию с системами контроля версий. В данном проекте использовался подход обновления развернутого приложения при каждом коммите, это позволило наискорейшим образом перезапускать обновлённое приложение в автоматическом режиме.

\end{document}
